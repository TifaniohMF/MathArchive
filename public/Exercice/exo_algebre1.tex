\documentclass[a4paper,12pt,french]{article}

% ====== IMPORTATION DES PACKAGES ======

\usepackage[T1]{fontenc}
\usepackage[utf8]{inputenc}
\usepackage{amsfonts}
\usepackage{lmodern}
\usepackage[margin=2cm]{geometry}
\usepackage[french]{babel}

\renewcommand{\familydefault}{\sfdefault}

\date{}
\author{}
\title{EXERCICES D'ALGEBRE 1}

\begin{document}
\maketitle

\section{Théorie Naïve des ensemble}
\subsection{Exercice} 
\begin{enumerate}
    \item Considérons les ensembles suivants : $A = \{1, 13, 25\} ; B = \{\{1, 13\} , 25\} ; C = \{\{1, 13, 25\}\} ; D = \{, 1, 13, 25\} ; E = \{25, 1, 13\} ; F = \{\{1, 13\} , \{25\}\} ; G = \{\{25\} , \{1, 13\} , 25\} ; H = \{\{1\} , \{13\} , 25\}.$
    \begin{itemize}
        \item [(a)] Quelles sont les relations (d’égalité ou d’inclusion) qui existent entre ces ensembles ?
        \item[(b)] Déterminer $A \cap B; \; G \cup H; \; E \backslash G; \; C_{D}^{A}$
    \end{itemize}
    \item Soient $A$, $B$ et $C$ trois parties d’un ensemble $E$ :
    \begin{itemize}
        \item[(a)] Montrer que :
                $$( A \cap B) \cup B^{c} = A \cup B^{c} \\ ( A \backslash B) \backslash C = A \backslash ( B \cap C ) \\A \backslash ( B \cap C ) = ( A \backslash B ) \cup ( A \backslash C ).$$
        \item[(b)]Simplifier : $( A \cup B)^{c} \cap (C \cup A^{c} )^{c} ;( A \cap B)^{c} \cup (C \cap A^{c} )^{c}$ .
    \end{itemize}
    \item Démontrer la Proposition 1.3.
\end{enumerate}

\subsection{Exercice}
Construisez des applications :
\begin{itemize}
    \item[-]Injective mais pas surjective ;
    \item[-]Surjective mais pas injective ;
    \item[-]Bijective ;
    \item[-]Ni injective ni surjective.
\end{itemize}

\subsection{Exercice}
$E = [0, 1] ; \; F = [-1, 1] ; \; G = [0, 2].$
Soient $f$ et $g$ deux applications définies respectivement par :
$$\begin{array}{ccc} f :E & \mapsto & G \\ x & \mapsto & 2 - x\end{array}; \begin{array}{ccc}g :F & \mapsto & G \\ x & \mapsto & x^{2} + 1\end{array}.$$
\begin{itemize}
    \item[(a)] Déterminer $f (\left\{ \frac{1}{2}\right\}), \; f^{-1}(\{0\}), \; g([1, -1]), \; g^{-1}([0,2])$
    \item[(b)] Les applications $f$ et $g$ sont-elles bijectives ? Justifier votre réponse.
\end{itemize}

\subsection{Exercice}
\begin{enumerate}
    \item Montrer que $\mathbb{Z}$ est dénombrable.
    \item Montrer que $\mathbb{N} \times \mathbb{N}$ est dénombrable. En déduire que le produit d’un nombre fini d’ensembles dénombrables est dénombrable.
    \item Montrer que $\mathbb{Q}$ est dénombrable.
    \item Soit $( E_{n} )_{n \in \mathbb{N}}$ une famille dénombrable de sous ensembles dénombrables d’un ensemble $E$. Montrer que la réunion $\cup_{n \in \mathbb{N}} E_{n}$ est dénombrable.
    \item Montrer que l’ensemble des polynômes à coefficients entiers est dénombrable. En déduire que l’ensemble des sous-ensembles finis de $\mathbb{N}$ est dénombrable.
    \item On dit qu’un nombre (réel ou complexe) est algébrique s’il est une racine d’un polynôme à coefficients entiers. Montrer que l’ensemble des nombres algébriques est dénombrable.
    \item Existe-il une bijection entre $\mathbb{Q} \cap [0, 1]$ et $\mathbb{Q} \cap  ] 0, 1 [ $ ? 
\end{enumerate}

\subsection{Exercice}
\begin{enumerate}
    \item En s’inspirant de la preuve du théorème 1.19, expliciter une bijection entre les intervalles $[ a, b[$ et $] a, b[$.
    \item Montrer que l’ensemble $\mathbb{N}^{\mathbb{N}}$ des suites d’entiers est équipotent à $\mathbb{R}$.
    \item Montrer que l’ensemble des parties de $\mathbb{R}$ n’est ni dénombrable, ni équipotent à $\mathbb{R}$.
    \item Montrer que l’ensemble $\mathbb{R}^{R}$ n’est ni dénombrable, ni équipotent à $\mathbb{R}$.
\end{enumerate}

\subsection{Exercice}
\begin{enumerate}
    \item Montrer que les relations suivantes sont des relations d’équivalences :
    \begin{itemize}
        \item[(i)] Le parallélisme sur l’ensemble des droites de $\mathbb{R}^{2}$ ou de $\mathbb{R}^{3}$;
        \item[(ii)] Sur $\mathbb{R}^{2}, \; (x, y)\mathcal{R}( x^{'}, y^{'})$ si et seulement si $x + y = x^{'} + y^{'}$.
    \end{itemize}
    \item Montrer que les relations suivantes sont des relations d’ordres partiels :
    \begin{itemize}
        \item[(i)] L’inclusion sur l’ensemble des parties $P(E)$ d’un ensemble $E$ ;
        \item[(ii)] La divisibilité sur l’ensemble des entiers $\mathbb{Z}$;
        \item[(iii)] Sur $\mathbb{R}^{2}, \; (x, y)\mathcal{T}( x^{'} , y^{'})$ si et seulement si $| x^{'} - x | \leq y^{'} - y$.
    \end{itemize}
    \item Soit $E = \mathbb{R}^{2} \backslash \{(0, 0)\}$. Considérons la relation binaire $\mathbb{R}$ sur $E$ définie comme suit : Pour tout $a$ et $b$ dans $E$, $a\mathcal{R}b$ si et seulement si $a$ et $b$ appartiennent à une droite passant par (0, 0).
    \begin{itemize}
        \item[(i)] Soient $(x, y)$ et $(x^{'}, y^{'})$ deux éléments de $E$. Montrer que $(x, y)\mathcal{R}( x^{'}, y^{'})$ si et seulement si il existe un nombre réel non nul $\lambda$ tel que $(x, y) = \lambda( x^{'}, y^{'})$.
        \item[(ii)] Montrer que $\mathcal{R}$ est une relation d’équivalence.
        \item[(iii)] Notons par $[x, y]$ la classe d’équivalence d’un élément $(x, y)$ de $E$. Vérifier qu’on a $[x, 1] = [y, 1]$ si et seulement si $x = y$.
        \item[(iv)] Montrer qu’on a : $E/\mathcal{R} = \{[ x, 1] : x \in \mathbb{R}\} \cup \{[1, 0]\}$
    \item (\textbf{Important.}) Soit $f$ une application d’un ensemble $E$ dans un ensemble $F$. On sait que la relation $\mathcal{R}$ définie pour tout $a$ et $b$ dans $E$, par : $$a\mathcal{R}b \Leftrightarrow f (a) = f(b)$$ est une relation d’équivalence.
    \end{itemize}
    \begin{itemize}
        \item[(i)] Montrer que l’application $\bar{f}$ de $E/ \mathcal{R}$ dans $F$ définie par $\bar{f} (\dot{a}) = f (a)$ est bien définie et est injective.
        \item[(ii)] En déduire qu’on a $f = \bar{f} \circ g$ où l’application $g$ est la projection canonique de $E$ dans $E/\mathbb{R}$.
        \item[(iii)] Montrer que si $f$ est surjective, alors il existe une bijection entre $E/ \mathbb{R}$ et $F$.
    \end{itemize}
\end{enumerate}

\section{Equation linéaire et matrice}
\subsection{Exercice}
\begin{enumerate}

    \item 
    \begin{itemize} 
        \item[(a)] Déterminez si le vecteur $(1, 3, 5) \in \mathbb{R}^{3}$ est une combinaison linéaire de $(0, 1, 0), (1, 4, 1) \mbox{ et } (1, 0, 1)$.
        \item[(b)] Déterminez si le vecteur $(1, 1) \in \mathbb{R}^{2}$ est une combinaison linéaire de $(0, 1), (1, 4)$ et $(1, 0)$. Dans le cas où la réponse est affirmative, est-ce que la représentation en tant que combinaison linéaire est unique ?
    \end{itemize}
    \item Décrivez le sous-ensemble de R3 formé par toutes les combinaisons linéaires des vecteurs $u = (1, 1, 0)$ et $v = (0, 1, 1)$. Trouvez un vecteur qui n’est pas combinaison linéaire de $u$ et $v$.
    \item Soient $u = (\pi, 0)$ et $v = (0, 2)$. Décrivez les sous ensembles de $\mathbb{R}^{2}$ suivants :
    \begin{itemize}
        \item[(a)]. $\{cu|c \in \mathbb{N}\}$.
        \item[(b)]. $\{cu|c \geq 0\}$.
        \item[(c)]. $\{cu + dv|c \in \mathbb{N} \mbox{ et } d \in \mathbb{R}\}$.
    \end{itemize}
    \item Est-ce que le vecteur $w = (1, 0)$ est une combinaison linéaire des vecteurs $u = (2, -1)$ et $v = (-1, 2)$ ?
    \item Si $u + v = ( 12 , 4, 1)$ et $u - 2v = (1, 0, 2)$, calculez $u$ et $v$.
    \item Montrez que pour tout vecteur $u$, $0u = 0$.
    \item Pour deux vecteurs $u$ et $v \in \mathbb{R}^{2}$, quand est-ce qu’on a l’égalité $|u \cdot v| = \|u\|v\|$ ? l’égalité $\|u + v\| = \|u\| + \|v\|$ ?
    \item Montrez que pour $z$, $w \in \mathbb{C}^{n}$ et $k \in \mathbb{K}$ on a :
    \begin{itemize}
        \item[(a)] $z \cdot w = w \cdot z$.
        \item[(b)] $(kz) \cdot w = z \cdot (kw)$.
        \item[(c)] $z \cdot (kw) = k (z \cdot w)$.
        (Comparer avel le cas réel).
    \end{itemize}
    \item
    \begin{itemize}
        \item[(a)] Soient $u = (a, b)$ et $v = (c, d)$ deux vecteurs du plan. Trouver une condition nécessaire et suffisante pour que tout élément de $\mathbb{R}^{2}$ soit une combinaison linéaire de $u$ et $v$.
        \item[(b)] Trouver quatre vecteurs de $\mathbb{R}^{4}$ tels que tout vecteur de $\mathbb{R}^{4}$ soit une combinaison linéaire de ces vecteurs.
    \end{itemize}
    \item Si $\|u\| = 5$ et $\|v\| = 3$, quelles sont la plus petite et la plus grande valeurs de $\|u - v\|$ ? Même question pour $u \cdot v$.
    \item Est-il possible d’avoir trois vecteurs du plan dont les produits scalaires (deux à deux) sont tous strictement négatifs ? Quand est-il dans $\mathbb{R}^{3}$ ?
    \item Soient $x$, $y$ et $z$ trois nombres réels tels que $x + y + z = 0$. Trouver l’angle que les vecteurs $u = (x, y, z)$ et $v = (z, x, y)$ font entre eux.
\end{enumerate}

\subsection{Exercice}
Dans toute la suite, sauf mention explicite du contraire, $\mathbb{K}$ désignera le corps des nombres réels $\mathbb{R}$ ou le corps des nombres complexes $\mathbb{C}$.
\begin{enumerate}
    \item Ecrire les deux problèmes suivants sous la forme $Ax = b$ où $A$ est une matrice $2 \times 2$, puis donner une solution à chaque problème :
        \begin{itemize}    
            \item[(a)] Alice est deux fois plus jeune que Bob et la somme de leur age est 33 ;
            \item[(b)] Les deux points $(2, 5)$ et $(3, 7)$ appartiennent à une droite d’équation $y = mx + c$. Trouver $m$ et $c$.
        \end{itemize}
    \item Pour chacune des matrices suivantes, trouver le scalaire $a$ pour que la matrice soit singulière (non inversible) : $$\left(\begin{array}{ccc} 1 & 3 & 5 \\1 & 2 & 4 \\ 1 & 1 & a \end{array} \right), \; \left(\begin{array}{ccc} 1 & 0 & a \\ 1 & 1 & 0 \\ o & 1 & 1 \end{array}\right), \;
    \left(\begin{array}{ccc}a & a & a \\ 2 & 1 & 5 \\ 3 & 3 & 6\end{array}\right)$$
    \item Soit $A$ une matrice dans $M_{3} (\mathbb{K})$ telle qu’il existe un vecteur colonne non nul $x$ dans $\mathbb{K}^{3}$ vérifiant $Ax = 0$.
        \begin{itemize}    
            \item[(a)] Montrer que les vecteurs colonnes de $A$ forment un plan $P$ dans $\mathbb{K}^{3}$ .
            \item[(b)] Montrer que $P$ et $x$ sont perpendiculaires.
        \end{itemize}
    \item Soit un système d’équations linéaires dans $\mathbb{K}^{3}$.
        \begin{itemize}
            \item[(a)] Montrer que ce système ne peut pas avoir exactement deux solutions.
            \item[(b)] Si $(x, y, z)$ et $(u, v, w)$ sont deux solutions du système, pouvez vous trouver un autre ?
        \end{itemize}
    \item Trouver les matrices matrices E et L telles que l’on ait : $$ EP_{3} = P_{2}, \; LP_{3} = I_{4}$$  où $$ P_{3} = \left(\begin{array}{cccc} 1 & 0 & 0 & 0 \\ 1 & 1 & 0 & 0 \\ 1 & 2 & 1 & 0 \\ 1 & 3 & 3 & 1 \end{array}\right), \;
    \left(\begin{array}{cccc} 1 & 0 & 0 & 0 \\ 0 & 1 & 0 & 0 \\ 0 & 1 & 1 & 0 \\ 0 & 1 & 2 & 1 \end{array}\right)$$
    \item considérons les matrices suivantes : $$ E_{1} = \left(\begin{array}{cccc} 1 & 0 & 0 & 0 \\ a & 1 & 0 & 0 \\ b & 0 & 1 & 0 \\ c & 0 & 0 & 1 \end{array}\right), \; E_{1} = \left(\begin{array}{cccc} 1 & 0 & 0 & 0 \\ 0 & 1 & 0 & 0 \\ 0 & d & 1 & 0 \\ 0 & e & 0 & 1 \end{array}\right), \;
    E_{3} = \left(\begin{array}{cccc} 1 & 0 & 0 & 0 \\ 0 & 1 & 0 & 0 \\ 0 & 0 & 1 & 0 \\ 0 & 0 & f & 1 \end{array}\right)$$ Montrons que ; $$ L = E_{1}E_{2}E_{3} = E_{1} = \left(\begin{array}{cccc} 1 & 0 & 0 & 0 \\ a & 1 & 0 & 0 \\ b & d & 1 & 0 \\ c & e & F & 1 \end{array}\right)$$
    \item Calculons les inverses des trois matrices suivantes : $$ A = \left(\begin{array}{cccc} 1 & -a & 0 & 0 \\ 0 & 1 & -b & 0 \\ 0 & 0 & 1 & -c \\ 0 & 0 & 0 & 1 \end{array}\right); \; B = \left(\begin{array}{cccc} 2 & -1 & 0 & 0 \\ -1 & 2 & -1 & 0 \\ 0 & -1 & 2 & -1 \\ 0 & 0 & -1 & 1 \end{array}\right); \;
    K = \left(\begin{array}{cccccc} 2 & -1 & 0 & 0 & 0 & 0\\ -1 & 2 & -1 & 0 & 0 & 0\\ 0 & -1 & 2 & -1 & 0 & 0\\ 0 & 0 & -1 & 2 & -1 & 0 \\ 0 & 0 & 0 & -1 & 2 & -1 \\ 0 & 0 & 0 & 0 & -1 & 2\end{array}\right)$$ Calculer $4K^{-1}$ et $7K^{-1}$.
    \item Soit $A$ et $B$ deux matrices carrées de même dimension.
        \begin{itemize}    
            \item[(a)] Montrer que $A( I + BA) = ( I + AB) A$.
            \item[(b)] En déduire que $I + BA$ est inversible si et seulement si $I + AB$ l’est aussi.
        \end{itemize}
    \item Un sous ensemble de $M_{n} (\mathbb{K})$ est appelé un groupe de matrices si pour toutes $A$ et $B$ deux matrices de l’ensemble, on a : le produit $AB$ et l’inverse de chaque élément sont dans l’ensemble.
        \begin{itemize}
            \item[(a)] Montrer que si $G$ est un groupe de matrices, la matrice identité $I_{n}$ est automatiquement dans $G$;
            \item[(b)] Montrer que : l’ensemble des matrices triangulaires inférieures telles que $a_{ii} = 1$; l’ensembles des matrices symétriques ; l’ensemble des matrices de permutations sont des groupes de matrices.
            \item[(c)] Donner plus de groupes de matrices.
        \end{itemize}
    \item Ecrire une matrice dans $M_{3} (\mathbb{K})$ de votre choix.
            \begin{itemize}
                \item[(a)] Trouver deux matrices $B$ et $C$ telles que : $A = B + C$ et $B$ et $C$ soient respectivement symétrique et anti-symétrique.
                \item[(b)] Ré-écrire $B$ et $C$ en fonction de $A$ et $A^{T}$.
            \end{itemize}
    \item Factoriser les matrices suivantes (de la forme $A = LU$ ou $PA = LU$) :
\end{enumerate}

\section{Espaces vectoriels et sous-espaces vectoriels}
\subsection{Exercice}
\begin{enumerate}
    \item Donnez un exemple montrant que la réunion de deux sous-espaces vectoriels n’est pas forcément un espace vectoriel.
    \item Soit $V = \mathbb{K}^{\mathbb{N}}$ l’ensemble de toutes les suites $( a_{1}, a_{2}, a_{3} , \cdots )$ d'éléments de $\mathbb{K}$ muni de l’addition par composante et de la multiplication par un scalaire par composante. Vérifiez que $V$ est un $\mathbb{K}$-espace vectoriel. Notons $\mathbb{K}^{(\mathbb{N})}$ le sous-ensemble des suites à support fini, i.e., les suites dont tous les termes sont nuls sauf un nombre fini d’entre eux. Montrez que $\mathbb{K}^{(\mathbb{N})}$ est un sous-espace vectoriel de $\mathbb{K}^{\mathbb{N}}$.
    \item Est-ce que le sous-ensemble de $\mathbb{R}^{2}$ suivant est un espace vectoriel sur $\mathbb{R}$ ? $$E = \{( x, y) \in \mathbb{R}^{2} | x \geq 0\}$$. Expliquez.
    \item Décidez, dans chacun des cas suivant, si $V = \mathbb{R}^{2}$ muni des lois d’additions et de multiplications par un scalaire données sont des espaces vectoriels sur R ou non (justifiez en cas de réponse négative) :
    \begin{itemize}
        \item[i)] $( a, b) + (c, d) = ( a + b, c + d)$ et $k ( a, b) = (ka, b)$.
        \item[ii)] $( a, b) + (c, d) = ( a + b)$ et $k ( a, b) = (ka, kb)$.
        \item[iii)] $( a, b) + (c, d) = ( a + b, c + d)$ et $k ( a, b) = (k^{2} a, k^{2} b)$.
    \end{itemize}
    \item Considérons le système d’equations linéaires en les inconnus $x_{1}, \cdots, x_{n}$ et à coefficients dans
    $\mathbb{R}$ suivant :
    $$\begin{array}{c}  a_{11} x_{1} + a_{12} x_{2} + \cdots + a_{1n} x_{n} = 0 \\
                    a_{21} x_{1} + a_{22} x_{2} + \cdots + a_{2n} x_{n} = 0 \\
                            \ldots
                    a_{m1} x_{1} + a_{m2} x_{2} + \cdots + a_{mn} x_{n} = 0
    \end{array}$$
    Un tel système est dit homogène (toutes les monômes ont le même degré, ici 1). Montrez que l’ensemble de toutes les solutions de ce système forme un sous-espace vectoriel de $\mathbb{R}^{n}$. Si le systéme linéaire n’est pas homogène, est-ce que l’ensemble des solutions forme toujours un espace vectoriel ? Si oui, démontrez, si non donnez un contre-exemple.
    

    \item Soit $V = F (E, \mathbb{K})$ comme dans Exemples 3.1.10 avec $\mathbb{K} = \mathbb{R}$. Montrez que le sous ensemble $W$ de $F ( E, \mathbb{K})$ des fonctions bornées dans V est un sous-espace vectoriel. On rappelle qu’une fonction $f$ á valeur réelle est bornée s’il existe un nombre réel $M \in \mathbb{R}$ tel que $| f x | \leq M$ pour tout $x \in E$.
    \item Donnez un système générateur de $\mathbb{C}^{2}$ en tant que $\mathbb{C}$-espace vectoriel, puis en tant que $\mathbb{R}$-espace vectoriel.
    \item Soit $V = \mathbb{R}^{4}$ en tant que $\mathbb{R}$-espace vectoriel. Déterminez si v = (3, 9, -4, -2) appartient au sous-espace de V engendré par $u_{1} = (1, -2, 0, 3), u2 = (2, 3, 0, -1)$ et $u_{3} = (2, -1, 2, 1)$.
    \item Décrivez les espaces colonnes des matrices suivantes :
    \begin{itemize}
        \item[a)] $ I_{2} = \left(\begin{array}{cc} 1 & 0 \\ 0 & 1\end{array}\right)$
        \item[b)] $A = \left(\begin{array}{cc} 2 & 3 \\ 1 & \frac{3}{2} \end{array}\right)$
        \item[c)] $A = \left(\begin{array}{ccc} 1 & 2 & 3 \\ 0 & 0 & 4 \end{array}\right)$
    \end{itemize}
    \item
    \begin{itemize}
        \item[a)] Décrivez un sous-espace de $M_{2} (\mathbb{R})$ contenant A = $ \left(\begin{array}{cc} 1 & 0 \\ 0 & 0\end{array}\right)$ mais pas B = $\left(\begin{array}{cc} 0 & 0 \\ 0 & -1\end{array}\right)$
        \item[b)] Si un sous-espace de $M_{2} (\mathbb{R})$ contient $A$ et $B$, est-ce que ce sous-espace doit contenir $I_{2}$ ?
        \item[c)] Décrivez un sous-espace de $M_{2} (\mathbb{R})$ ne contenant aucune matrice diagonale non-nulle.
        Une matrice carée est dite diagonale si tous ses coefficients qui ne sont pas sur la diagonale sont nuls.
    \end{itemize}
    \item Vrai ou faux, justifiez votre réponse :
    \begin{itemize}
        \item[a)] L’ensemble des matrices symétriques de $M_{n} (\mathbb{K})$ forme un sous-espace vectoriel.
        \item[b)] L’ensemble des matrices antisymétriques de $M_{n} (\mathbb{K})$ forme un sous-espace vectoriel.
        \item[c)] L’ensemble des matrices de $M_{n} (\mathbb{K})$ qui ne sont pas symétriques forme un sous-espace vectoriel.
    \end{itemize}
    \item Vrai ou faux, justifiez votre réponse :
    \begin{itemize}
        \item[a)] Les éléments $b$ qui ne sont pas dans $C(A)$ (pour une matrice $A$) forme un sous-espace.
        \item[b)] Si $C(A) = 0$, alors $A$ est la matrice nulle.
        \item[c)] Soit $A \in M_{m \times n} (\mathbb{R})$, alors $C(2A) = C(A)$.
        \item[d)] Soit $A \in M_{n} (\mathbb{K})$, alors $C(A - I_{n}) = C(A)$
    \end{itemize}
\end{enumerate}  

\subsection{Exercice}
\begin{enumerate}
    \item Considérons $\mathbb{R}^{4}$ en tant que $\mathbb{R}$-espace vectoriel. Montrez que les vecteurs suivants sont indépendants : $(6, 2, 3, 4), (0, 5, -3, 1)$ et $(0, 0, 7, -2)$.
    \item Montrez que deux vecteurs sont linéairement dépendants si et seulement si l’un d’entre eux est un multiple de l’autre.
    \item Donnez une base du $\mathbb{R}$-espace vectoriel $V = \{( a, a, b)| a, b \in \mathbb{R}\}$.
    \item Soit le $\mathbb{Q}$-espace vectoriel $V = \mathbb{Q}^{3}$ et $W \subseteq V$ le sous-espace engendré par $E = \{(1, 2, 3), (2, 3, 4), (3, 5, 7)\}$. Montrez que $E$ n’est pas une base de $V$ mais que $\{(1, 2, 3), (2, 3, 4)\}$ en est une.
    \item Montrez que l’ensemble suivant est un R-espace vectoriel : $V = \{ f : \mathbb{N} \rightarrow \mathbb{R} | \exists S \subseteq \mathbb{N} \mbox{ fini, pour tout } n \in \mathbb{N} \backslash S, f (n) = 0\}$. Trouvez une base de $V$.
\end{enumerate}

\section{Théorie des groupes et homomorphisme des groupes}
\subsection{Exercice}
\begin{enumerate}
    \item Pour chacune des opérations binaires sur Z suivantes laquelle est associative ? laquelle est commutative ? laquelle admet un identité ?
    \begin{itemize}
        \item[i)] $(x, y) \mapsto x - y$;
        \item[ii)] $(x, y) \mapsto x y$;
        \item[iii)] $x \star y:= xy - x - y + 2$.
    \end{itemize}
    \item Considérons le rectangle R défini par $(x, y) \in \mathbb{R}^{2} : | x | \leq n; |y| \leq m$ où $n$ et $m$ sont des entiers naturels non nuls distincts. Montrer que les quatre symétries de R suivantes forment un groupe avec la loi de composition des applications :
    \begin{itemize}
        \item[a)] L’identité $e : (x, y) \mapsto (x, y)$;
        \item[b)] La réflexion $r_{1} : (x, y) \mapsto (x, -y)$;
        \item[c)] La réflexion $r_{2} : (x, y) \mapsto (- x, y)$;
        \item[d)] La rotation d’angle $\pi$ et de centre $(0, 0) r_{3} : (x, y) \mapsto (- x, -y)$. Ecrire le table de Cayley de ce groupe. Le groupe est-il abélien ?
    \end{itemize}
    \item Pour chacune des structures suivantes, déterminer si c’est un groupe ou pas :
    \begin{itemize}
        \item[i)] $(\mathbb{Z}, -)$ où l’opération binaire - désigne la soustraction usuelle sur les nombres ;
        \item[ii)] $(\mathbb{R}, \star)$ où l’opération binaire $\star$ est définie par : $x \star y:= x + y - 1$;
        \item[iii)] $\mathbb{Z}/n\mathbb{Z} \backslash \{0\} , \times$ où n est un nombre entier naturel non nul ;
        \item[iv)] $(\{z \in \mathbb{C} : |z| = 1\} , \times)$ où $\times$ désigne la multiplication de nombres complexes.
    \end{itemize}
    \item Compléter les tables de Cayley des groupes suivants : $$\begin{array}{|c|c|c|c|} \hline  \star  & e & a & b \\ \hline e & & a & \\ \hline a & & & \\ \hline b & & & \\ \hline \end{array} \; \begin{array}{|c|c|c|c|c|} \hline  \star & a & b & c & d \\ \hline a & & d & & \\ \hline b & a & & & \\ \hline c & & & & \\ \hline d & & & b & \\ \hline\end{array}$$
    \item Montrer que les sous-ensembles suivants ne sont pas de sous-groupes de $(\mathbb{Z}, +)$ : L’ensemble des nombres entiers impairs ; L’ensemble des entiers positifs ; L’ensemble $\{-2, -1, 0, 1, 2\}$; L’ensemble vide.
    \item Déterminer les ordres des groupes suivants : $(\mathbb{Z}/n\mathbb{Z}, +); (\mathbb{Z}, +); S_{n} ; GL_{2} (\mathbb{R}).$
    \item Déterminer les ordres des éléments suivants :
    \begin{itemize}
        \item[i)] $ 2 \in (\mathbb{Z}/10 \mathbb{Z}, +)$;
        \item[ii)] $\left(\begin{array}{cc} 0 & -1 \\ 1 & 0 \end{array}\right) \in GL_{n}(\mathbb{R})$
        \item[iii)] L’identité dans un groupe ;
        \item[iv)] $\pi \in (\mathbb{R}, +)$;
    \end{itemize}
    \item Soient $(G, \star)$ un groupe fini d’identité $e$ et $g \in G$. Montrer qu’ils existent deux entiers naturels non nuls distincts $i$ et $j$ tels que $a_{i} = a_{j}$. En déduire qu’il existe un entier naturel non nul $n$ tel que $a_{n} = e$.
    \item Soit $(G, \star)$ un groupe d’ordre pair. Montrer qu’il existe un élément $a$ de $G$ tel que $a_{2} = e$.
    \item Déterminer tout les sous-groupes des groupes suivants : $(\mathbb{Z}/6\mathbb{Z}, +)$ et $(\mathbb{Z}/12\mathbb{Z}, +)$.
    \item Montrer que les sous-groupes de $(\mathbb{Z}, +)$ sont de la forme $n\mathbb{Z}$ où n est un entier naturel.
    \item Soient $G$ un groupe fini et $g$ un élément de $G$. Montrer que $|g|$ divise $|G|$.
\end{enumerate}

\subsection{Exercice}
\begin{enumerate}
    \item Pour chacune des applications suivantes, déterminer si c’est un homomorphisme de groupes ou pas :
    \begin{itemize}
        \item[i)] $(\mathbb{R} \backslash \{0\} , \times)  \rightarrow (\mathbb{R}, +), \; a \mapsto log|a|$;
        \item[ii)] $(\mathbb{C}, +) \rightarrow (R, +), \; x \mapsto |x|$;
        \item[iii)] $(\mathbb{Z}, +) \rightarrow (\mathbb{Z}, +), \; t \mapsto 5t$;
        \item[iv)] $(\mathbb{R}^{2} , +) \rightarrow (R, +), \; (x,y) \mapsto xy$;
        \item[v)] $(\mathbb{R}^{2} , +)  \rightarrow (\mathbb{R}, +), \; (u,v) \mapsto u - v$.
    \end{itemize}
    \item Soient $( G_{1} , \star_{1} )$ et $( G_{2} , \star_{2} )$ deux groupes. Montrer que $G_{1} \times G_{2} \simeq G_{2} \times G_{1}$.
    \item Considérons les groupes $G_{1} = (\mathbb{Z}/2\mathbb{Z} \times \mathbb{Z}/2\mathbb{Z}, +)$ et $G_{2} = (\mathbb{Z}/4\mathbb{Z}, +)$. Montrer que $G_{1} \not\simeq G_{2}$.
    \item Considérons l’homomorphisme de groupes $f : (\mathbb{Z}/12\mathbb{Z}, +) \rightarrow (\mathbb{C}^{\star}, \times) \; n \mapsto i^{n}$, où $i$ est un élément de $\mathbb{C}$ tel que $i^{2} = -1$. Déterminez $Im( f )$ et $Ker( f )$ ainsi que $(\mathbb{Z}/12\mathbb{Z})/Ker( f )$. Puis écrire les tables de Cayley des groupes $(\mathbb{Z}/12\mathbb{Z})/Ker( f )$ et $Im( f )$. Vérifier qu’ils sont bien similaires.
    \item Pour chacune des homomorphismes suivantes, déterminer son image et son noyau. Puis écrire le résultat qu’on obtient du premier théorème d’isomorphisme :
    \begin{itemize}
        \item[i)] $(\mathbb{Z}, +) \rightarrow (\mathbb{Z}, +), \; m \mapsto 2m$;
        \item[ii)] $(\mathbb{R}, +) \rightarrow (\mathbb{C}^{\star} , \times). \; x \mapsto e^{ix}$;
        \item[iii)] $det : (GL_{2} (\mathbb{R}) \rightarrow (\mathbb{C}^{\star} , \times))$;
        \item[iv)] $f : \mathbb{Z} \rightarrow \mathbb{Z}/2\mathbb{Z} \times \mathbb{Z}/3\mathbb{Z}$;
    \end{itemize}
    \item Soit $G$ un groupe abélien fini tel que pour tout élément $x$ de $G$, on a $x = e$ ou $x_{2} = e$ où $e$ est l’identité de $G$. Montrer que $| G | = 2n$ où $n$ est un entier naturel.
    \item Soient $G$ un groupe et $H$ un sous groupe d’indice 2 de $G$. Montrer que $H$ est un sous groupe normal de $G$.
\end{enumerate}

\subsection{Exercice}
\begin{enumerate}
    \item Pour chacune des permutations suivantes, écrire-la comme composée des cycles:
    \begin{itemize}
        \item[i)] $\left(\begin{array}{cccc} 1 & 2 & 3 & 4 \\ 2 & 1 & 4 & 3 \end{array}\right)$;
        \item[ii)] $\left(\begin{array}{cccccc} 1 & 2 & 3 & 4 & 5 & 6 \\ 4 & 1 & 5 & 3 & 6 & 2 \end{array}\right)$;
        \item[iii)] $\left(\begin{array}{cccccc} 1 & 2 & 3 & 4 & 5 & 6 \\ 1 & 6 & 4 & 3 & 2 & 5 \end{array}\right)$.
    \end{itemize}
    \item Pour chacune des permutations suivantes, écrire-la en matrice-notation :
    \begin{itemize}
        \item[i)] $(1 \; 2 \;  3)(4 \; 6 \; 8)$;
        \item[ii)] $(1 \; 6)(4 \; 2)(5 \; 3)$;
        \item[iii)] $(1 \; 5 \; 3)$;
        \item[iv)] $(1 \; 9 \; 3)(2 \; 6)(7 \; 8)$;
        \item[v)] $(2 \; 4)(3 \; 5 \; 7).$
    \end{itemize}
    \item Pour chacune des cas suivants, calculer $\sigma \tau$ et $\tau \sigma$ :
    \begin{itemize}
        \item[i)] Dans $S_{4}$, $\sigma = \left(\begin{array}{cccc} 1 & 2 & 3 & 4 \\ 2 & 3 & 4 & 1 \end{array}\right),\;\tau = (1 \; 2)(3 \; 4);$
        \item[ii)] Dans $S_{5}$. $\sigma = \left(\begin{array}{ccccc} 1 & 2 & 3 & 4 & 5\\ 5 & 4 & 3 & 2 & 1\end{array}\right), \; \tau = \left(\begin{array}{ccccc} 1 & 2 & 3 & 4 & 5\\ 1 & 3 & 5 & 4 & 2 \end{array}\right);$
        \item[iii)] Dans $S_{6}$, $\sigma = (1 \; 2)(5 \; 6), \; \tau = (1 \; 3 \; 4 \; 6 \; 2)$.
        Ecrire les résultats en utilisant les deux notations.
    \end{itemize}
    \item Pour chacune des permutations suivantes, déterminer si elle est paire ou impaire :
    \begin{itemize}
        \item[i)] $\left(\begin{array}{ccccc} 1 & 2 & 3 & 4 & 5 \\ 2 & 5 & 3 & 1 & 4\end{array}\right)$;
        \item[ii)] $\left(\begin{array}{ccccc} 1 & 2 & 3 & 4 & 5 \\ 2 & 3 & 5 & 1 & 4\end{array}\right)$;
        \item[iii)] $\left(\begin{array}{cccccc} 1 & 2 & 3 & 4 & 5 & 6 \\ 6 & 5 & 4 & 3 & 2 & 1 \end{array}\right)$.
    \end{itemize}
    \item
    \begin{itemize}
        \item[i)] Respectivement dans $S_{3}$ et $S_{4}$, combien de transpositions a t-on ? Quand est-il des 3-cycles ? des 4-cycles (dans S4 ) ?
        \item[ii)] Soit $H$ l’ensemble des éléments qui ne sont pas des transpositions, ni des 3-cycles, nides 4-cycles dans S4 . Le sous-ensemble $H$ est-il un sous groupe ? Si oui, déterminer sonordre et trouver un groupe usuel (qu’on connait très bien) qui est isomorphe à $H$.
    \end{itemize}
    \item Dans chacun des groupes suivants, déterminer le nombre d’éléments d’ordre 2 et le nombre d’éléments d’ordre 3 : $S_{3}$, $A_{4}, D_{5}$ et $D_{6}$.
    \item Pour tout $k \in \{1, 2, 3, 4, 5, 6\}$, déterminer le nombre d’éléments d’ordre $k$ dans $A_{5}$.
    \item Pour chacun des cas suivants, déterminer si le sous-ensemble est un sous-groupe de $S_{4}$ ou pas. Dans le cas où le sous-ensemble est un groupe, déterminer si c’est un groupe normal :
    \begin{itemize}
        \item[i)] $\{ id, (1 \; 3 \; 4), (1 \; 4 \; 3)\}$ ;
        \item[ii)] $\{ id, (1 \; 2)(3 \; 4), (1 \; 3)(2 \; 4), (1 \; 4)(2 \; 3)\}$ ;
        \item[iii)] $\{ id,  (1 \; 2 \; 3 \; 4), (1 \; 4 \; 3 \; 2), (1 \; 3)(2 \; 4)\}$ ;
        \item[iv)] $\{ id,  (1 \; 2 \; 3), (1 \; 3 \; 2), (2 \; 3 \; 4), (2 \; 4 \; 3)\}$.
    \end{itemize}
    \item Considérons l’application $f : (\mathbb{Z} \times \mathbb{Z}) \rightarrow S_{4}, \; (m, n) \mapsto \sigma^{n} \tau^{m}$ où $\sigma = (1 \; 2)(3 \; 4)$ et $\tau = (1 \; 3)(2 \; 4)$. Montrer que $f$ est un homomorphisme de groupes. Expliciter le résultat u premier théorème d’isomorphisme.
    \item Notons respectivement par $\sigma$ et $\tau$ une rotation et une réflexion dans le groupe diédral $D_{n}$. Montrer que $\tau^{-1} = \rho \tau \rho$.
    \item Considérons un homomorphisme $f : S_{n} \rightarrow \mathbb{Z}/3\mathbb{Z}$ où $n \geq 3$. Montrer que $f$ est trivial,i.e, pour tout $\sigma \in S_{3}$, on a $f (\sigma) = 0$.
    \item 
    \begin{itemize}
        \item[i)] Montrer que $S_{3} \simeq D_{3}$.
        \item[ii)] Expliciter tout les éléments de $D_{3}$ en forme de matrices, en utilisant le fait qu’il est isomorphe à un sous groupe de $O_{2} (\mathbb{R})$.
    \end{itemize}
    \item
    \begin{itemize}
        \item[i)] Soit $(G, \star)$ un groupe d’ordre un nombre premier $p$. Montrer que $G \simeq \mathbb{Z}/p\mathbb{Z}$, i.e, à isomorphisme près, il n’existe qu’un seul groupe d’ordre $p$ où $p$ un nombre premier.
        \item[ii)] Pour tout $n \in \{1, 2, 3, 4, 5, 6, 7\}$ , trouver les classes d’isomorphismes des groupes d’ordre n.
    \end{itemize}
\end{enumerate}
\end{document}

